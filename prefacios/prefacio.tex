\chapter*{}
%\thispagestyle{empty}
%\cleardoublepage

%\thispagestyle{empty}

\input{portada/portada_2}



\cleardoublepage
\thispagestyle{empty}

\begin{center}
{\large\bfseries Algoritmos socioinspirados: implementación, estudio y comparativa}\\
\end{center}
\begin{center}
Juan José Sierra González\\
\end{center}

%\vspace{0.7cm}
\noindent{\textbf{Palabras clave}: algoritmo, metaheurística, sociedad, socioinspirado, comparativa, análisis experimental...}\\

\vspace{0.7cm}
\noindent{\textbf{Resumen}}\\

Los algoritmos socioinspirados son un tipo especial de metaheurística que basa su comportamiento en la sociedad humana. Suponen un sencillo acercamiento al complejo mundo de la computación evolutiva, gracias a sus múltiples símiles con patrones reconocibles del día a día. El objetivo de este estudio es comprobar si, además de sus propiedades didácticas, estos algoritmos presentan una buena capacidad de resolución de problemas de optimización. Mediante un minucioso análisis experimental se recogerán los resultados de estos algoritmos para un conocido benchmark y se compararán con los de algunos algoritmos de referencia que permitan valorar adecuadamente su potencial en este tipo de problemas.
\cleardoublepage


\thispagestyle{empty}


\begin{center}
{\large\bfseries Socioinspired algorithms: implementation, study and comparative analysis}\\
\end{center}
\begin{center}
Juan José Sierra González\\
\end{center}

%\vspace{0.7cm}
\noindent{\textbf{Keywords}: algorithm, metaheuristic, society, socioinspired, comparative analysis, experimental analysis...}\\

\vspace{0.7cm}
\noindent{\textbf{Abstract}}\\

Socioinspired algorithms are a special kind of metaheuristic that base their functioning on human society. They are studied as a simple approach to the complex world of evolutionary computation, thanks to their multiple similarities with everyday life recognizable patterns. The main goal of this study is to check if, besides their teaching properties, these algorithms are able to perform well at solving optimization problems. Results gathered with a meticulous experimental analysis will be compared to some other reference algorithms that will allow guessing the potential in this kind of problems.

\chapter*{}
\thispagestyle{empty}

\noindent\rule[-1ex]{\textwidth}{2pt}\\[4.5ex]

Yo, \textbf{Juan José Sierra González}, alumno de la titulación Grado en Ingeniería Informática de la \textbf{Escuela Técnica Superior
de Ingenierías Informática y de Telecomunicación de la Universidad de Granada}, con DNI 76589592Y, autorizo la
ubicación de la siguiente copia de mi Trabajo Fin de Grado en la biblioteca del centro para que pueda ser
consultada por las personas que lo deseen.

\vspace{6cm}

\noindent Fdo: Juan José Sierra González

\vspace{2cm}

\begin{flushright}
Granada a 7 de septiembre de 2018.
\end{flushright}


\chapter*{}
\thispagestyle{empty}

\noindent\rule[-1ex]{\textwidth}{2pt}\\[4.5ex]

D. \textbf{Daniel Molina Cabrera}, Profesor del Área de Soft Computing and Intelligent Information Systems del Departamento de Ciencias de la Computación e Inteligencia Artificial de la Universidad de Granada.

\vspace{0.5cm}

\textbf{Informa:}

\vspace{0.5cm}

Que el presente trabajo, titulado \textit{\textbf{Algoritmos socioinspirados: implementación, estudio y comparativa}},
ha sido realizado bajo su supervisión por \textbf{Juan José Sierra González}, y autoriza la defensa de dicho trabajo ante el tribunal
que corresponda.

\vspace{0.5cm}

Y para que conste, expide y firma el presente informe en Granada a 7 de septiembre de 2018.

\vspace{1cm}

\textbf{El director:}

\vspace{5cm}

\noindent \textbf{Daniel Molina Cabrera}

\chapter*{Agradecimientos}
\thispagestyle{empty}

       \vspace{1cm}


A mi familia, en especial a mi madre y a mi hermana, por su infinita paciencia conmigo.\\

A mis amigos y compañeros informáticos, junto a los que he crecido como informático pero sobre todo como persona.\\

A mi pareja, por todo su apoyo y afecto durante el proceso.\\

Y por último, pero no menos importante, a mi tutor, por ayudarme a realizar este trabajo.\\