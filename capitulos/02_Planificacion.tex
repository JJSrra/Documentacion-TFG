\chapter{Planificación}

En este capítulo se aborda el plan de trabajo a seguir para la realización de este estudio. En primer lugar se estimarán los requisitos a satisfacer por el mismo, a fin de lograr los objetivos planteados, y a continuación se planteará su planificación, como los presupuestos necesarios, la carga de trabajo por fase de dicho estudio o los tiempos esperados de realización para cada una de ellas.

\section{Requisitos de investigación}

Al tratarse de un trabajo de investigación, un análisis de requisitos habitual no puede ser aplicado correctamente a esta situación. En su lugar, sin embargo, se proponen una serie de objetivos a cumplir para la conclusión del estudio. En el caso de este en particular, los distintos requisitos que se pueden distinguir son:

\begin{enumerate}
	\item{\textbf{Realizar una investigación primeriza acerca de los algoritmos socioinspirados}: revisar la literatura buscando información sobre lo que representan estos algoritmos, valorar las motivaciones que impulsen a realizar el estudio y obtener propuestas de dichos algoritmos.}
	\item{\textbf{Seleccionar las propuestas más interesantes}: de entre todos los algoritmos socioinspirados que se hayan podido encontrar en la fase anterior, seleccionar aquellos que resulten más interesantes o que aporten un enfoque diferente al panorama actual. También se busca que pertenezcan a campos distintos dentro de esta rama de algoritmos, a fin de aportar un punto de vista sobre las principales vertientes que existen.}
	\item{\textbf{Analizar a fondo las propuestas seleccionadas}: el principal objetivo aquí es ser capaz de entender qué metodología sigue cada algoritmo, en qué carga teórica basa sus técnicas y qué es capaz de conseguir con lo que propone. Se intentará asemejar cada propuesta con otros algoritmos evolutivos que sean más reconocibles por cualquier investigador iniciado en el campo.}
	\item{\textbf{Implementar aquellos algoritmos de los que no se posea código fuente}: ya que el análisis de este trabajo es experimental, es necesario contar con el código de los algoritmos para poder realizar adecuadamente las distintas pruebas. Una implementación propia facilita a su vez que se pueda adaptar el código de dicho algoritmo para seguir unas pautas de formato de soluciones comunes a todo el estudio.}
	\item{\textbf{Estimar los parámetros de los algoritmos}: si se desconocen los parámetros ideales con los que ejecutar un algoritmo, se someterá a una pequeña experimentación con varias combinaciones de parámetros a fin de seleccionar los que mejores resultados aporten.}
	\item{\textbf{Realizar la experimentación}: utilizar los algoritmos implementados y un benchmark de referencia para obtener resultados. Los algoritmos se lanzarán con las combinaciones de parámetros extraídas del apartado anterior, en función de las conclusiones obtenidas.}
	\item{\textbf{Construcción de tablas y gráficas experimentales}: en base a los resultados obtenidos, se pueden disponer los datos en tablas representativas, así como en gráficas de convergencia con las que comprobar el comportamiento de cada algoritmo a lo largo de las ejecuciones.}
	\item{\textbf{Estudio analítico de los datos obtenidos}: comparar los resultados de cada algoritmo con los de un algoritmo evolutivo de referencia, así como entre ellos, a fin de descubrir cómo se comportan con una serie variada de funciones complejas y qué algoritmos destacan más sobre los otros.}
	\item{\textbf{Extraer conclusiones del estudio realizado}: dar una visión analítica de la eficacia de los algoritmos socioinspirados, basándose en la comparativa con el algoritmo de referencia, y justificar qué propuestas son más prometedoras.}
	\item{\textbf{Trabajos futuros a realizar}: valorar en qué aspectos se puede innovar en este campo, aportar propuestas de mejora para las técnicas socioinspiradas y realizar un ajuste minucioso de parámetros para los algoritmos con mayor potencial.}
\end{enumerate}

Con los requisitos planteados, la planificación del trabajo debe abarcar cada uno de esos pasos y estimar un tiempo a priori con el que se pueda solventar cada requisito. Además, debe abarcar todo el material e infraestructura necesarios y, junto al tiempo estimado, dar una idea del presupuesto que requiere realizar este estudio.