\chapter{Introducción}

La informática ha evolucionado mucho desde sus comienzos, y resolver complejas fórmulas matemáticas o computar cálculos de grandes dimensiones ya no es un problema. Existen multitud de algoritmos que se encargan de realizar estas tareas, y que cualquier usuario puede utilizar sin tener un conocimiento experto. Los nuevos retos de la informática están plagados de problemas que no tienen una solución clara y precisa, entre ellos los \textbf{problemas de optimización}, que se estudiarán en este trabajo. Así como un problema de cálculo está bien limitado y un programador puede idear un algoritmo que, en base a distintas fórmulas y operaciones, obtenga siempre el resultado deseado, no se puede decir lo mismo de los problemas de optimización, en los que influyen multitud de factores y que no tienen una solución absoluta.

Esta situación se pone de manifiesto cuando queremos obtener los valores que hacen mínima una función, y su dominio es tan grande que probar con el rudimentario método de fuerza bruta no es una opción. Para esto se diseñaron las \textbf{metaheurísticas}, algoritmos que buscan la mejor solución a una función y que no necesitan ser planteados de forma diferente dependiendo de cada una. A lo largo de su (aún breve) historia han existido ya multitud de algoritmos diferentes que aportan su solución a este problema, pero este trabajo se centrará en las nuevas metaheurísticas \textbf{socioinspiradas} y analizará lo que aportan al panorama actual y sobre qué se fundamentan.

\section{Motivación}

Las metaheurísticas, y en general la computación evolutiva, están a la orden del día en el ámbito de la investigación. Cada año se publican un gran número de nuevas propuestas y técnicas y muchas de ellas ven la luz en numerosos congresos en todo el mundo. En algunos de ellos incluso se realizan competiciones entre las nuevas propuestas, como pasa con el Congress of Evolutionary Computation (CEC).

Con estas competiciones se busca incentivar a los investigadores a mejorar las soluciones actuales constantemente. En la actualidad, el auge de la ciencia de datos y del manejo de flujo de datos en tiempo real, y con ello de lo que conocemos como Big Data, premia que estos algoritmos estén muy optimizados y sean capaces de dar buenos resultados ya no sólo en términos de efectividad, sino de rapidez.

Las metaheurísticas que más éxito han tenido en este campo han sido habitualmente las \textbf{bioinspiradas}, es decir, aquellas que basan su funcionamiento en la naturaleza, y generalmente en animales. Tras el éxito cosechado por dichos algoritmos, nacen nuevas versiones que, tratando de emular a los anteriormente citados, realizan sus operaciones siguiendo un modelo basado en comportamientos de la sociedad. Estos algoritmos son denominados \textbf{socioinspirados}. Aportan un diseño interesante, innovador y más sencillo de comprender y de aplicar que los algoritmos evolutivos tradicionales, y son el centro del estudio de este trabajo.

\section{Objetivos}

El principal objetivo de este Trabajo de Fin de Grado es analizar cómo funcionan algunas de las versiones más interesantes de algoritmos socioinspirados, mediante un estudio comparativo con veinticinco funciones de un benchmark. 

Se estudiarán un total de \textbf{seis algoritmos socioinspirados}, a saber:

\begin{itemize}
	\item \textbf{Soccer League Competition (SLC)} \cite{slc-article}
	\item \textbf{Imperialist Competitive Algorithm (ICA)} \cite{ica-conference}
	\item \textbf{Parliamentary Optimization Algorithm (POA)} \cite{poa-article}
	\item \textbf{Social Emotional Optimization Algorithm (SEA)} \cite{sea-chapter}
	\item \textbf{Anarchic Society Optimization Algorithm (ASO)} \cite{aso-article} \cite{aso-chapter}
	\item \textbf{Ideology Algorithm (IA)} \cite{ia-article}
\end{itemize}

El objetivo global y final fruto del estudio de este trabajo es analizar cómo de efectivas son estas nuevas técnicas y si tienen un hueco entre la élite de la computación evolutiva, así como lo encontraron sus análogos bioinspirados. Para ello, en primer lugar se ha de \textbf{realizar un estudio en la literatura} del algoritmo, a fin de comprender al completo cómo se comporta el mismo en cada momento del proceso de optimización. El conocimiento obtenido de cada uno será documentado en su correspondiente capítulo de la memoria.

Una vez comprendido todo el proceso, tiene lugar la fase de \textbf{implementación}, basada en los distintos \textit{papers} que se han encontrado para cada algoritmo. Para ello se han utilizado los conocimientos adquiridos en la fase anterior, siendo lo más fiel posible al diseño planteado. Más información acerca de la implementación se detallará en el capítulo homónimo.

Con todos los algoritmos implementados, el siguiente paso, y más importante de cara al estudio a realizar, es la \textbf{experimentación}. Durante esta fase se enfrentará a cada uno de estos algoritmos a un benchmark de variadas funciones que evalúen apropiadamente cómo se comporta en una buena muestra de ejemplos.

Finalmente, el propio estudio en sí pasa por analizar los resultados obtenidos para cada algoritmo y función y compararlos de dos formas diferentes. En primer lugar, se incluirá dentro del capítulo de experimentación de cada uno una comparativa con un ya reconocido \textbf{algoritmo de referencia}, a fin de asegurar si dicho socioinspirado puede optar a una solución al menos tan buena como las estándares. En segundo lugar, y para finalizar el estudio, se compararán los resultados de \textbf{todos los algoritmos socioinspirados entre sí}, se analizarán pros y contras de las mejores soluciones y se dará una perspectiva global de la posición de estos algoritmos en el mundo de las metaheurísticas.