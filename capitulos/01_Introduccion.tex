\chapter{Introducción}

La informática ha evolucionado mucho desde sus comienzos, y resolver complejas fórmulas matemáticas o computar cálculos de grandes dimensiones ya no es un problema. Existen multitud de algoritmos que se encargan de realizar estas tareas, y que cualquier usuario puede utilizar sin tener un conocimiento experto. Los nuevos retos de la informática están plagados de problemas que no tienen una solución clara y precisa, entre ellos los \textbf{problemas de optimización}, que se estudiarán en este trabajo. Así como un problema de cálculo está bien limitado y un programador puede idear un algoritmo que, en base a distintas fórmulas y operaciones, obtenga el resultado deseado, no se puede decir lo mismo de los problemas de optimización, en los que influyen multitud de factores y no tienen una solución absoluta.\\

Esta situación se pone de manifiesto cuando queremos obtener los valores que hacen mínima una función, y su dominio es tan grande que probar con el rudimentario método de fuerza bruta no es una opción. Para esto se diseñaron las \textbf{metaheurísticas}, algoritmos que buscan la mejor solución a una función y que no necesitan ser planteados de forma diferente para cada una. A lo largo de su (aún breve) historia han existido ya multitud de algoritmos diferentes que aportan su solución a este problema, pero en este trabajo me centraré en las nuevas metaheurísticas \textbf{socioinspiradas} y analizaré lo que aportan al panorama actual y sobre qué se fundamentan.