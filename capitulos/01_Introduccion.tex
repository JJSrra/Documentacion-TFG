\chapter{Introducción}

La informática ha evolucionado mucho desde sus comienzos, y resolver complejas fórmulas matemáticas o computar cálculos de grandes dimensiones ya no es un problema. Existen multitud de algoritmos que se encargan de realizar estas tareas, y que cualquier usuario puede utilizar sin tener un conocimiento experto. Los nuevos retos de la informática están plagados de problemas que no tienen una solución clara y precisa, entre ellos los \textbf{problemas de optimización}, que se estudiarán en este trabajo. Así como un problema de cálculo está bien limitado y un programador puede idear un algoritmo que, en base a distintas fórmulas y operaciones, obtenga siempre el resultado deseado, no se puede decir lo mismo de los problemas de optimización, en los que influyen multitud de factores y que no tienen una solución absoluta.\\

Esta situación se pone de manifiesto cuando queremos obtener los valores que hacen mínima una función, y su dominio es tan grande que probar con el rudimentario método de fuerza bruta no es una opción. Para esto se diseñaron las \textbf{metaheurísticas}, algoritmos que buscan la mejor solución a una función y que no necesitan ser planteados de forma diferente dependiendo de cada una. A lo largo de su (aún breve) historia han existido ya multitud de algoritmos diferentes que aportan su solución a este problema, pero en este trabajo me centraré en las nuevas metaheurísticas \textbf{socioinspiradas} y analizaré lo que aportan al panorama actual y sobre qué se fundamentan.

\section{Motivación}

Las metaheurísticas, y en general la computación evolutiva, están a la orden del día en el ámbito de la investigación. Cada año se publican un gran número de nuevas propuestas y muchas de ellas ven la luz en numerosos congresos en todo el mundo. En algunos de ellos incluso se realizan competiciones entre las nuevas propuestas, como pasa con el Congress of Evolutionary Computation (CEC).\\

Con estas competiciones se busca incentivar a los investigadores a mejorar las soluciones actuales constantemente. En la actualidad, el auge de la ciencia de datos y del manejo de flujo de datos en tiempo real, y con ello de lo que conocemos como Big Data, premia que estos algoritmos estén muy optimizados y sean capaces de dar buenos resultados ya no sólo en términos de efectividad, sino de rapidez.\\

Las metaheurísticas que más éxito han tenido en este campo han sido habitualmente las \textbf{bioinspiradas}, es decir, aquellas que basan su funcionamiento en la naturaleza, y generalmente en animales. Tras el éxito cosechado por dichos algoritmos, nacen nuevas versiones que, tratando de emular a los anteriormente citados, realizan sus operaciones siguiendo un modelo basado en comportamientos de la sociedad. Estos algoritmos son denominados \textbf{socioinspirados}. Aportan un diseño interesante e innovador y son el centro del estudio de este trabajo.

\section{Objetivos}

