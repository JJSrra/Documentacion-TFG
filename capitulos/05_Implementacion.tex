\chapter{Implementación y análisis de algoritmos}

En este apartado se mostrarán los detalles más destacados de la implementación de cada algoritmo estudiado, comentando sus particularidades y resumiendo lo que ha supuesto esta fase en cuanto al desarrollo completo del proyecto.

\section{Detalles generales sobre la implementación}

En primer lugar, antes de comenzar con la explicación de los algoritmos, conviene dar una visión general acerca de la implementación en este estudio, cómo se ha llevado a cabo y con qué herramientas.

Los algoritmos se han implementado utilizando el lenguaje de programación \textbf{Python} en su versión 3.6 \cite{python-3.6-doc}, elección personal del alumno, dada su facilidad para trabajar con vectores y matrices utilizando librerías como \textbf{Numpy} \cite{numpy-doc}, además de otras de representación gráfica como \textbf{Matplotlib} \cite{matplotlib-doc}. Poder operar de forma sencilla con múltiples matrices que habitualmente constituirán la población de soluciones de cada problema resultó un detalle determinante para tomar la decisión.

Encontrar el código fuente de los algoritmos mencionados resulta una ardua tarea, ya que la mayoría de los autores no liberan su código en ninguna plataforma. En algunas ocasiones es posible encontrar versiones en el lenguaje de programación \textbf{MATLAB}, uno de los lenguajes más utilizados en el ámbito científico y de investigación por su capacidad de cómputo y de operar con fórmulas matemáticas de gran dimensión. Sin embargo, MATLAB es un software privativo y de pago, y el alumno ha querido liberar su proyecto desde el primer momento, por lo que se rechazó la idea de implementarlos en este lenguaje. Sin embargo, sí que se ha utilizado como referencia para algún algoritmo una versión de MATLAB sobre la que basar la implementación en Python.

En relación a lo comentado en el párrafo anterior, el proyecto ha estado libre y alojado en la plataforma \textbf{GitHub} \cite{repositorio-tfg} desde que se comenzó la implementación, para que cualquier usuario interesado pudiese no sólo acceder a él sino también comprobar cómo está hecho, y en un futuro incluso realizar los cambios pertinentes que considerase, cumpliendo así el paradigma del software libre.

\section{Imperialist Competitive Algorithm (ICA)}
