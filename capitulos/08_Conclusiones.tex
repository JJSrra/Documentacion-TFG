\chapter{Conclusiones y posibles extensiones}

A lo largo de todo este estudio se han estudiado una serie de propuestas de algoritmos socioinspirados, a fin de arrojar algo de luz sobre un campo de la algorítmica del que aún no se tienen muchos resultados. Estos algoritmos tienen un interés muy obvio en la relación de algoritmos evolutivos con patrones cotidianos que son sencillos de entender y de aplicar. Pero lo que aún no se ha comprobado es si realmente este tipo de algoritmos está preparado para afrontar un benchmark de una competición y vencer a otras propuestas más conocidas.

El estudio realizado ha abarcado todas las fases de una experimentación completa, desde la revisión de la literatura hasta la experimentación y el análisis resultante, pasando por un previo análisis teórico que permitiese entender y valorar cada uno correctamente.

En el apartado del análisis experimental se ha llegado a la conclusión de que los algoritmos basados en una agrupación de PSOs, como ICA y POA, son menos robustos y tienden a dar resultados peores que los que van un paso más allá y modifican aún más el comportamiento estándar de un PSO. En este grupo puede entrar el algoritmo IA, que explora más posiciones en cada iteración y por tanto tiene más margen de maniobra para llegar a una mejor solución. En el último escalón de los socioinspirados basados en una estructura PSO se encontrarían los algoritmos SEA y ASO, dos técnicas que utilizan la memoria histórica en el problema para tomar decisiones de desplazamiento en base a ella. Gracias a esto, la mejor posición no tiene por qué estar necesariamente siempre presente en la población, lo que da lugar a que haya un movimiento más variado y se produzca una mayor exploración en el espacio de búsqueda que evite estancarse pronto en óptimos locales. Finalmente, el algoritmo socioinspirado que mejor resultado ha dado no es una variación de un PSO sino que recuerda más a un algoritmo genético, o a un Differential Evolution, y se trata del SLC.

No obstante, el hecho de que el SLC haya arrasado dentro del estudio no implica que sea un algoritmo que refleje buenos resultados. A pesar de que todos los algoritmos, en mayor o menor medida, han terminado venciendo al PSO, el algoritmo que se ha utilizado como referencia, este no supone un reto real ya que no se trata de ninguna versión pulida. Al enfrentar los resultados del SLC frente a un algoritmo pulido de hace 10 años, los resultados han quedado bien expuestos: ni siquiera el mejor de los algoritmos socioinspirados es suficientemente bueno como para ser digno de participar en una competición. Al menos, es así por el momento, hasta que una nueva propuesta lo desmienta.

Y es que realmente la motivación que rodea a estos algoritmos no es tanto computacional sino didáctica. La complejidad de un algoritmo evolutivo queda reducida sustancialmente cuando se explica desde un contexto socioinspirado, y como se ha notado a lo largo del estudio, estos algoritmos ni siquiera son difíciles de entender aun no teniendo base conceptual sobre el tema. Valorando esto, y desde el punto de vista de la enseñanza, desde este estudio se anima a que sean utilizados para ayudar a comprender los algoritmos más complejos que puedan ser un rompecabezas de primeras.

Como posibles extensiones a este trabajo, en primer lugar cabría la posibilidad de comparar los algoritmos socioinspirados con nuevas variantes de PSO que hay surgido con el paso de los años, para hacer una estimación de en qué punto se encuentran. Más adelante, dependiendo de cómo resulten ser estas comparativas, se puede hacer un nuevo estudio para determinar qué características de los algoritmos que mejor resultado reflejen se pueden aplicar sobre el resto de algoritmos, y valorar su eficacia. Aunque, por supuesto, será mejor si todo tiene una base socioinspirada a la que aferrarse.